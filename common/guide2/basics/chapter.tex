\section{Basics}
While \src{Common} is a layer atop of \src{Core}, \src{Common} itself consists of three more layers: \src{common}, \src{facile} and \src{support} (in their respective packages). The \src{facile} layer mostly contains stand-alone abstractions of classes/interfaces of \src{Core}, the \src{common} layer brings these abstractions together. The \src{support} layer contains exactly what it's name suggest: small, generic classes and methods that do not fit anywhere but that are really helpful in building up the other layers.

Clients almost exclusively have to make use of the \src{common} layer. They can use the other layers, but it seldomly makes sense to do so.

\subsection{Concepts}
In the understanding of \src{Common} an application consists of one main-window and maybe several supportive frames and dialogs. The main-window is most times a \src{JFrame} and the application runs as long as this frame is visible. The main-window consists of several panels, each showing some part of the data. E.g. the panels of a web-browser could be the ``history'', the ``bookmarks'' and the open websites.

\src{Common} adds an additional layer between panels and main-frame, it separates them and allows the user to drag \& drop panels. For this to happen the client needs to wrap each panel into a \src{CDockable}. These \src{CDockable}s are put onto a set of \src{CStation}s, a controller (of type \src{CControl}) manages the look, position, behavior etc. of all these elements. 

\begin{figure}[ht]
\centering
\includegraphics[scale=1]{basics/app_without}
\caption{The standard application without \src{Common}. A main-frame and some panels that are put onto the main-frame.}
\label{fig:app_without}
\end{figure}

\begin{figure}[ht]
\centering
\includegraphics[scale=1]{basics/app_with}
\caption{An application with \src{Common}. The panels are wrapped into \src{dockable}s. The \src{dockable}s are put onto stations which lay on the main-frame. \src{Dockable}s can be moved to different stations.}
\label{fig:app_with}
\end{figure}

\subsection{Hello World}
A first example containing only three colored panels will introduce the very basic vocabulary. In depth discussions of the concepts and implementations follow in the chapters afterwards.

\subsubsection{Setup controller}
The first step should be to create a \src{CControl}. This central controller wires all the objects of the framework together. A \src{CControl} needs to know the root window of the application, it is used as parent for any dialog that may be opened (e.g. during a drag \& drop operation a dialog may be used to paint the dragged element). Most applications will be able to just forward their root window to one of the constructors.

The code to create the controller looks like this:
\begin{lstlisting}
public class Example{
	public static void main( String[] args ){
		JFrame frame = new JFrame();
		frame.setDefaultCloseOperation( JFrame.EXIT_ON_CLOSE );
		
		CControl control = new CControl( frame );
		
		...
\end{lstlisting}

\subsubsection{Setup stations}
The second step is to setup the layer between main-frame and \src{dockable}s. There are different \src{CStation}s available, for example the \src{CMinimizeArea} shows minimized \src{CDockable}s. But most applications will always use the same layout: some station in the center of the frame shows a grid of \src{CDockable}s and on the four edges minimized \src{CDockable}s are listed. The class \src{CContentArea} is a combination of several \src{CStation}s offers exactly that layout.

There is always a default-\src{CContentArea} available, it can be accessed by calling \src{getContentArea} of \src{CControl}. If required additional \src{CContentArea}s can be created by the method \src{createContentArea} of \src{CControl}.

A \src{CContentArea} is a \src{JComponent}, so its usage is straight forward. Line \src{10} is the important new line in this code:
\begin{lstlisting}
public class Example{
	public static void main( String[] args ){
		JFrame frame = new JFrame();
		
		frame.setDefaultCloseOperation( JFrame.EXIT_ON_CLOSE );
		
		CControl control = new CControl( frame );
		
		frame.setLayout( new GridLayout( 1, 1 ) );
		frame.add( control.getContentArea() );
		
		...
\end{lstlisting}

\infobox{\src{CControl} always creates an additional station for handling free floating \src{CDockable}s.}

\subsubsection{Setup dockables}
The last step is to set up some \src{CDockable}s. \src{CDockable}s are the things that can be dragged and dropped by the user. A \src{CDockable} has a set of properties, e.g. what text to show as title, whether it can be maximized, what font to use when focused, and so on.

\src{CDockable} is just an interface and clients should always use one of the two subclasses \src{DefaultSingleCDockable} or \src{DefaultMultipleCDockable}.Without going into details: \src{single-dockable}s exist exactly once, while \linebreak \src{multi-dockable}s can be created and destroyed by the framework anytime.

In the code below new single \src{dockable}s are created in lines \src{23-25} and \src{43-48}. They need to be registered at the \src{CControl} in lines \src{27-29}, otherwise they cannot be shown. Optionally the initial location can be set like in line \src{33} and \src{36}. The initial location is applied in the moment when the \src{dockable} gets visible, it will not have any influence afterwards. So there is no point in setting the location of the first \src{dockable}, since there are no other \src{dockable}s it gets all the space anyway and the initial location does not matter afterwards. With other words: the order in which \src{dockable}s are made visible is important.

\infobox{There is a class \src{CGrid} which allows to build an initial layout more easily, more about locations can be found in chapter \ref{sec:location}}

\begin{lstlisting}
import java.awt.Color;
import java.awt.GridLayout;

import javax.swing.JFrame;
import javax.swing.JPanel;

import bibliothek.gui.dock.common.CControl;
import bibliothek.gui.dock.common.CLocation;
import bibliothek.gui.dock.common.DefaultSingleCDockable;
import bibliothek.gui.dock.common.SingleCDockable;

public class Example{
	public static void main( String[] args ){
		JFrame frame = new JFrame();
		
		frame.setDefaultCloseOperation( JFrame.EXIT_ON_CLOSE );
		
		CControl control = new CControl( frame );
		
		frame.setLayout( new GridLayout( 1, 1 ) );
		frame.add( control.getContentArea() );
		
		SingleCDockable red = create( "Red", Color.RED );
		SingleCDockable green = create( "Green", Color.GREEN );
		SingleCDockable blue = create( "Blue", Color.BLUE );
		
		control.add( red );
		control.add( green );
		control.add( blue );
		
		red.setVisible( true );
		
		green.setLocation( CLocation.base().normalSouth( 0.4 ));
		green.setVisible( true );
		
		blue.setLocation( CLocation.base().normalEast( 0.3 ) );
		blue.setVisible( true );
		
		frame.setBounds( 20, 20, 400, 400 );
		frame.setVisible( true );
	}
	
	public static SingleCDockable create( String title, Color color ){
		JPanel background = new JPanel();
		background.setOpaque( true );
		background.setBackground( color );
		
		return new DefaultSingleCDockable( title, title, background );
	}
}

\end{lstlisting} 
