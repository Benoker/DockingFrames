\section{Introduction}

\src{DockingFrames} is an open source Java Swing framework. This framework allows to write applications with floating panels: \src{Component}s that can be moved around by the user.

\src{DockingFrames} consists of two libraries, \src{Core} and \src{Common}. \src{Common} provides advanced functionalities that are built on top of \src{Core}, it is a wrapper around \src{Core} and requires \src{Core} to work.

This guide does not claim to be complete nor that all of its parts are relevant. It is intended as a starting point to explain basic concepts and to find out which classes, interfaces and properties are important for developers. This document only covers \src{Common}, \src{Core} has its own guide. 

You can utilize \src{Common} without understanding \src{Core}, but knowing at least some basics about \src{Core} will make life much easier.
