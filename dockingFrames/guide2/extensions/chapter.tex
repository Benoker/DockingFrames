\section{Extensions} \label{sec:extensions}
Extensions allow libraries to add new code to the framework, this code will be treated as if it were always part of the framework. Basically it is a plug-in mechanism. Currently there are not many points where an extension can be inserted, new extension-points will be added when needed.

\src{Extension}s are collected by the \src{ExtensionManager}. Any module can call \src{load} to load extensions that match some \src{ExtensionName}. 

\designbox{Extensions were introduced in 1.0.8 to allow the usage of the glass-components. The glass-components could not be added directly to the framework due to licencing issues.} 

\subsubsection{Glass Extension}
The glass-extension adds new icons and a new way to paint tabs to the \src{EclipseTheme}. Clients only need to ensure that the libraries \src{extensionGlass.jar} and \src{glasslib.jar} are part of the classpath. The \src{ExtensionManager} will then automatically load this extension.

\warningbox{The Glass Extension is licensed by a modified version of the LGPL. You are prohibited to use the library if your application provides ``pornography, racialistics, violence, or the like material''. 

(A note by the author of this document: This licence may prohibit a photo-editing tool, an email-client used by atheists, or any software used by the police.)
}

\subsubsection{Extension Points}
A number of extensions exists. The following list only includes the extensions of the \src{Core} library.

\begin{description}
 \extension{ChoiceExtension}{ChoiceExtension.CHOICE\_EXTENSION}{Allows to add additional entries to a \src{Choice}. A \src{Choice} is a preference allowing the user to pick one of many items.}
 \extension{DockThemeExtension}{DockThemeExtension.DOCK\_THEME\_EXTENSION}{Allows to modify a \src{DockTheme} during the installation process.}
 \extension{ColorScheme}{ColorScheme.EXTENSION\_NAME}{Allows to extend or override the contents of a \src{ColorScheme}.}
\end{description}