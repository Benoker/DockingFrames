\documentclass[a4paper,10pt]{article}
\usepackage{listings}
\usepackage{color}

\lstset{language=Java}
\lstset{breaklines=true, numbers=left}
\lstset{tabsize=4}

\definecolor{CommentColor}{rgb}{0,0.5,0} 
\definecolor{KeywordColor}{rgb}{0,0,0.5}

\lstset{commentstyle=\scriptsize\color{CommentColor}\itshape}
\lstset{keywordstyle=\scriptsize\color{KeywordColor}\bfseries}
\lstset{basicstyle=\scriptsize}
\lstset{identifierstyle=\scriptsize}
\lstset{stringstyle=\scriptsize}

% \lstset{basicstyle=\ttfamily}

\newcommand{\src}[1]{\lstinline[basicstyle=\normalsize\ttfamily,keywordstyle=\normalsize\ttfamily,identifierstyle=\normalsize\ttfamily]|#1|}

\title{DockingFrames 1.0.2 - FAQ}
\author{Benjamin Sigg}

\begin{document}

\maketitle
\tableofcontents
\newpage

\begin{abstract}Some questions that are frequently asked.
\end{abstract}

\section{Writing applications}
This section deals with questions that are related to writing code.
\subsection{How to write a close-button?}
A "close-button" is some button, most times a cross in the upper right edge, that closes a \src{Dockable} when pressed.
\subsubsection{Solution 1: FDockable}
When using the common-project, then you can use \src{DefaultFSingleDockable} or \src{DefaultFMultipleDockable}. Both \src{FDockable}s contain a method \src{setCloseable}. Call that method with \src{true}.
\begin{lstlisting}
DefaultFSingleDockable dockable = ...
dockable.setCloseable( true );
\end{lstlisting}
If you implement the interface \src{FSingleDockable} or \src{FMultipleDockable} directly, then ensure that \src{isCloseable} returns \src{true}.

\subsubsection{Solution 2: DockFrontend}
When using a \src{DockFrontend}, register the \src{Dockable} at the \src{DockFrontend} and call \src{setHideable}.
\begin{lstlisting}
DockFrontend frontend = ...
Dockable dockable = ...

frontend.add( dockable, "a unique id" );
frontend.setHideable( dockable, true );
\end{lstlisting}

\subsubsection{Solution 3: CloseAction}
If you use the common-project, then a \src{DockAction} called \src{CloseAction} is available. Add the action to the \src{Dockable}s which should be closeable, for example you could use an \src{ActionGuard}:
\begin{lstlisting}
public class CloseGuard implements ActionGuard{
	private DockActionSource source;
    
	public CloseGuard( DockController controller ){
		DockAction close = new CloseAction( controller );
		source = new DefaultDockActionSource(
			new LocationHint(
				LocationHint.ACTION_GUARD,
				LocationHint.RIGHT_OF_ALL ),
			close );
	}

	public boolean react( Dockable dockable ) {
		return true;
	}
	public DockActionSource getSource( Dockable dockable ) {
		return source;
	}
}
\end{lstlisting}
And later:
\begin{lstlisting}
DockController controller = ...
controller.addActionGuard( new CloseGuard( controller ));
\end{lstlisting}

\subsubsection{Solution 4: New DockAction}
If \src{CloseAction} of the common-project can't be used, then a new \src{DockAction} must be written.
\begin{lstlisting}
public class CloseAction extends SimpleButtonAction{
	public CloseAction( DockController controller ){
		setText( "Close" );
		setIcon( new ImageIcon( "close.png" ));
	}

	@Override
	public void action( Dockable dockable ) {
		DockStation parent = dockable.getDockParent();
		if( parent != null )
			parent.drag( dockable );
	}
}
\end{lstlisting}
This action is then used as described in the third solution.

\subsection{How do I layout the contents of a SplitDockStation?}
The \src{SplitDockStation} has a complex layout. How can new \src{Dockable}s be added to \src{SplitDockStation} such that they have a certain location and size?

\subsubsection{Solution 1: SplitDockProperty}
A \src{SplitDockProperty} describes the size and location of a \src{Dockable} on a \src{SplitDockStation} by storing the \src{x}, \src{y} coordinates, and the \src{width} and \src{height}. All the properties are normalized such that they are between 0 and 1.

A client can create new \src{SplitDockProperty}s and call \src{drop}:
\begin{lstlisting}
SplitDockStation station = ...
Dockable alpha = ...
Dockable beta = ...

if( !station.drop( alpha, SplitDockProperty.NORTH ) )
	station.drop( alpha );

if( !station.drop( beta, new SplitDockProperty( 0, 0, 1, 1 )))
	station.drop( beta );
\end{lstlisting}
A few words to this code: in lines \src{5} and \src{8}, the result of \src{drop} needs to be checked. It is possible, that \src{SplitDockStation} refuses to add a \src{Dockable}.

\src{SplitDockStation} internally has a binary tree in whose leafs the \src{Dockable}s are stored, and the nodes determine the proportions between the \src{Dockable}s. Each \src{drop} adds a new branch into that tree. The \src{SplitDockProperty} is only a hint where to insert the branch, and will not be stored for later use. So this station does \textbf{not} work like a \src{LayoutManager}, the order in which the \src{Dockable}s are dropped is important. The first \src{Dockable} will always get boundaries of \src{0,0,1,1}.

\subsubsection{Solution 2: SplitDockPathProperty}
As set earlier, \src{SplitDockStation} is internally organized as a binary tree. A \src{SplitDockPathProperty} is the description of the exact location of a branch. It is used like \src{SplitDockProperty}:
\begin{lstlisting}
SplitDockStation station = ...
Dockable dockable = ...

SplitDockPathProperty path = new SplitDockPathProperty();
path.add( SplitDockPathProperty.Location.BOTTOM, 0.4 );
path.add( SplitDockPathProperty.Location.LEFT, 0.25 );
if( !station.drop( dockable, path ))
	station.drop( dockable );
\end{lstlisting}
In lines \src{5,6} a branch to the bottom left edge is created.

\subsubsection{Solution 3: SplitDockTree}
A \src{SplitDockTree} is an exact model of the internal binary tree that every \src{SplitDockStation} has. When calling \src{dropTree}, all \src{Dockable}s will be removed from the station, and the new tree will replace the old one.
\begin{lstlisting}
SplitDockStation station = ...
Dockable[] dockables = ...

SplitDockTree tree = new SplitDockTree();
tree.root(
	tree.horizontal(
		tree.vertical( 
			dockables[0], 
			dockables[1], 
			0.4 ),
		tree.vertical(
			tree.put(
				dockables[2] ),
			tree.put(
				dockables[3],
				dockables[4] ) )
	)
);
station.dropTree( tree );
\end{lstlisting}
Note that several \src{Dockable}s can be put at the same location as shown in lines \src{14-16}.

\subsubsection{Solution 4: SplitDockGrid}
A \src{SplitDockGrid} is an algorithm that takes several \src{Dockable}s and their desired location, and creates a \src{SplitDockTree} that matches these locations as good as possible.

The use is straight forward:
\begin{lstlisting}
SplitDockStation station = ...
Dockable[] dockables = ...

SplitDockGrid grid = new SplitDockGrid();
grid.addDockable( 0.0, 0.0, 0.5, 0.5, dockables[0] );
grid.addDockable( 0.0, 0.5, 0.5, 0.5, dockables[1] );
grid.addDockable( 0.5, 0.0, 0.5, 0.5, dockables[2] );
grid.addDockable( 0.5, 0.5, 0.5, 0.5, dockables[3] );
station.dropTree( grid.toTree() );
\end{lstlisting}

There is also the possibility to model the layout with a \src{String}:
\begin{lstlisting}
SplitDockStation station = ...
Dockable[] dockables = ...

Map<String, Dockable[]> map = new HashMap<String, Dockable[]>();
for( int i = 0; i < dockables.length; i++ ){
	map.put( 
		String.valueOf( i ), 
		new Dockable[]{ dockables[i] });
}

String layout = 
	"0022\n"+
	"0022\n"+
	"1133\n"+
	"1133";

SplitDockGrid grid = new SplitDockGrid( layout, map );
station.dropTree( grid.toTree() );
\end{lstlisting}
The layout is defined in lines \src{12-15}, just image a raster where the characters tell which \src{Dockable} should overlap a given cell.

\subsubsection{Solution 5: Start and store}
If your application is able to store the layout, then just start the application, make the layout by hand, and store the layout.

The common-project or \src{DockFrontend} can help you storing one or several layouts.

\end{document}
