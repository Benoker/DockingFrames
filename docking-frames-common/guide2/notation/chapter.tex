\section{Notation}
This document uses various notations.

Any element that can be source code (e.g. a class name) and project names are written mono-spaced like this: \src{java.lang.String}. The package of classes and interfaces is rarely given since almost no name is used twice. The packages can be easily found with the help of the generated API documentation (\src{JavaDoc}).

\infobox{Tips and tricks are listed in boxes.}

\warningbox{Important notes and warnings are listed in boxes like this one.}

\classbox{Implementation details, especially lists of class names, are written in boxes like this.}

\designbox{These boxes explain \textit{why} something was designed the way it is. This might either contain some bit of history or an explanation why some awkward design is not as bad as it first looks.} 

\codebox{References to examples illustrating something are marked with these boxes. All examples are stored in the ``tutorial'' project that is downloaded together with the Core and Common library.}